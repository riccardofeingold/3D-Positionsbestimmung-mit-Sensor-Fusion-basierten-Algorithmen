\documentclass{scrreprt}

% Schmale Seitenränder festlegen
\KOMAoption{DIV}{15}

% Vordefinierte Texte (z.B. "Inhaltsverzeichnis") auf Deutsch
\usepackage[ngerman]{babel}

% Unterstützung von UTF-8 (Unicode)
\usepackage[utf8]{inputenc}

% Unterstützung der modernen Zeichencodierung
\usepackage[T1]{fontenc}

% Moderne Schriftart verwenden
\usepackage{lmodern}

% Schriftart Georgia verwenden
\usepackage[T1]{fontenc}
\fontfamily{Georgia}\selectfont

% serifenfreie Schriftvariante verwenden
\renewcommand{\familydefault}{\rmdefault}

% Einbinden von Pixelgrafiken (JPG, PNG)
\usepackage{graphicx}

% Einbinden von Vektorgrafiken (SVG)
\usepackage{svg}

% Kopf- und Fusszeilen selbst definieren
\usepackage{scrlayer-scrpage}

% Voraussetzung für biblatex
\usepackage{csquotes}

% Zitate und Bibliografie
\usepackage[style=ieee]{biblatex}

%Bibliographie hinzufügen 
\addbibresource{Bibliographie/Bibliographie.bib}

\title{3D Positionsbestimmung mit Hilfe von Sensor-Fusion basierenden Algorithmen}
\subtitle{Entwicklung eines Embedded-System zur Bestimmung der 3D Position mit Arduino}
\author{Riccardo Orion Feingold\vspace{1cm}\\Betreut durch \\Stefan Rothe}

\publishers{\includesvg{bilder/logo}\vspace{1cm}\\Gymnasium Kirchenfeld\\Abteilung MN}
\subject{Maturaarbeit}

% selbstdefinierte Kopf- und Fusszeile verwenden
\pagestyle{scrheadings}

\lohead{Maturaarbeit}
\rohead{Riccardo Orion Feingold}
\cofoot{\thepage}

% Trennlinie für Kopfzeile einschalten
% \KOMAoption{headsepline}{on}

% Trennlinie für Fusszeile einschalten
% KOMAoption{footsepline}{on}

\begin{document}

\maketitle

\tableofcontents
\input{Texte/Vorwort.tex}
\chapter{Abstrakt}
\textbf{
In this paper a location estimation algorithm is briefly presented - especially focused on the vertical position. The idea is based on the combination of a Map-Matching algorithm and a Kalman Filter. Both are implemented in a altimeter, which consists out of a GPS shield and a sensor. While the GPS is giving the x,y and z coordinates as well as the velocity, the sensor can measure the temperature, pressure and the altitude over mean sea level (MSL) by using the so called international height formulae (or another one). The goal of this paper is to introduce some basic knowledge and to give a short answer on to the question: How can the vehicle’s positioning be improved?
}

\chapter{Einleitung}
In den letzten Jahren hat die Positionsbestimmung an Wichtigkeit zugenommen, da immer mehr Verkehrsmittel genaue Positionsdaten benötigen. Sie erlauben nicht nur eine Standort-Abfrage, sondern aus ihnen lassen sich auch Ankunftszeiten an einem gewissen Ort im Voraus bestimmen oder u.a. verlorene Smartphones wiederfinden. Doch sie werden in der Zukunft weiter an Wichtigkeit zunehmen. Zum Beispiel wird sie für selbstfahrende bzw. selbstfliegende Verkehrsmittel von enormer Bedeutung sein, vor allem für das letztgenannte. Denn hier müssen drei Koordinaten genau sein: Einerseits die X und Y Koordinaten, und die Höhe. Letzteres wirft die häufigsten Probleme auf, denn da treten die meisten Abweichungen in der Messung auf, zum Beispiel beim GPS (Globale Positioning System) oder Altimeter (Barometer, das die Höhe mit Hilfe einer Höhenformel berechnet) - die beiden einfachsten und häufig genutzten Varianten. Kombiniert man beide vernünftig, so dass sie sich gegenseitig unterstützen, lässt sich ein genaueres Höhenmessgerät entwickeln.

So ist das Ziel der Arbeit ein Embedded-System zur Höhenmessung zu entwickeln, welches nahezu in jeder Situation, bei denen GPS und Altimeter ungenaue Resultate liefern würden, zuverlässige Messungen garantiert. Dabei wird nicht nur ein Hardware Konzept geschaffen, sondern auch eine dazu passende Software geschrieben. Ebenfalls Gegenstand der Arbeit ist das Testen unterschiedlicher Arten der Höhenmessung auf ihre Stärken und Schwächen. 

Um dieses Embedded-System bestehend aus einem GPS-Modul, einem Barometer und einem Beschleunigungssensor zu entwickeln, beruft sich der Autor auf ein Vorgehen, welches man unter den Namen “Sensor Fusion” kennt. Darunter versteht man das Kombinieren mehrerer Sensoren, mit dem Ziel bessere Resultate zu erhalten als mit einem einzelnen Sensor. Dieses Prinzip wird mit Hilfe zweier Methoden umgesetzt. Bei einer handelt es sich um das sogenannte Map-Matching (siehe Kapitel 3.1) und bei der anderen um den Kalman-Filter (siehe Kapitel 3.2). In Kapitel 4 werden die einzelnen Schritte im Rahmen einer Prozessdokumentation erläutert. 

%Noch Ändern
Zunächst werden einige theoretische Grundlagen zur Höhenmessung bekannt gegeben, die für das Verständnis der kommenden Probleme wichtig sind - ein solches Problem wäre z.B.: Warum gibt das GPS unpräzise Höhendaten in einem Parkhaus? Nebst dem GPS werden auch der Altimeter und der Beschleunigungssensor als weitere Varianten zur Höhenmessung erklärt. Danach werden die zwei Methoden - Map-Matching und Kalman-Filter - erläutert, sodass die benötigten Grundlagen für das nächste Kapitel gelegt sind. Dieses wäre die Prozessdokumentation des Produktes. Im Anschluss wird der Autor kurz auf die Resultate der Messungen eingehen und im Diskussionsteil interpretieren. Sprich, erklären warum die Resultate so aussehen und welche Bestandteile noch fehlen, damit es zu einem besseren Ergebnis kommen könnte.

\chapter{Höhenmessung}
Die Höhenmessung - ein Teilgebiet der Geodäsie, sowie teilweise auch der Topografie und der Ingenieurs- und Bautechnik - wird in vielfältigen Bereichen wie z.B. bei der Raumplanung, beim Bergsteigen oder bei der Navigation verwendet \parencite{"Wikipedia:Höhenmessung"} . 

%Altimeter
\section{Altimeter}

\subsection{Höhenformel}

\subsection{Funktionsprinzip eines einfachen Barometers}

\subsection{BME280 Sensor} 

%GPS
\section{GPS}

\subsection{Wie funktioniert die Positionsbestimmung?}

\subsection{Effizienz der vertikalen Positionsbestimmung}

\subsection{Adafruit GPS Sensor}

%Möglichkeit die Höhe mit dem Beschleunigungssensor zu messen
\section{Die Höhe mit dem Beschleunigungssensor messen?}


%Accelerometer
\chapter{Inertial Measurement Unit Sensor}

\section{Beschleunigungssenor}

\section{Gyroscope}

\section{Magnetometer}

%\subsection{Wie lässt sich die Höhe mit einem Beschleunigungssensor berechnen?}

\section{BNO055}

\section{Vergleich}

%Sonstiges Material
\chapter{User Interface mit LCD und Keypad}

%Methoden 
\chapter{Methoden für die Umsetzung von Sensor Fusion}

\section{Map-Matching}

\subsection{Point-to-Point Methode}

\subsection{1. Implementation}

\subsection{2. Implementation}

\subsection{Probleme}

\section{Kalman-Filter}

\subsection{Kalman-Equation}

\subsection{Ein Höhenmodell?}

\subsection{Probleme}

\section{Complementary-Filter}

\subsection{Probleme}
\input{Texte/Resultate.tex}
\input{Texte/Diskussion.tex}

\listoffigures
%\printbibliography

\end{document}
