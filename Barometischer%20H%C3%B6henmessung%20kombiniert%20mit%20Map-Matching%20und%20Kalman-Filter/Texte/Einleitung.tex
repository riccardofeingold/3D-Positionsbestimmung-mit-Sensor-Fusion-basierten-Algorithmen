\chapter{Einleitung}
In den letzten Jahren hat die Positionsbestimmung an Wichtigkeit zugenommen, da immer mehr Verkehrsmittel genaue Positionsdaten benötigen. Sie erlauben nicht nur eine Standort-Abfrage, sondern aus ihnen lassen sich auch Ankunftszeiten an einem gewissen Ort im Voraus bestimmen oder u.a. verlorene Smartphones wiederfinden. Doch sie werden in der Zukunft weiter an Wichtigkeit zunehmen. Zum Beispiel wird sie für selbstfahrende bzw. selbstfliegende Verkehrsmittel von enormer Bedeutung sein, vor allem für das letztgenannte. Denn hier müssen drei Koordinaten genau sein: Einerseits die X und Y Koordinaten, und die Höhe. Letzteres wirft die häufigsten Probleme auf, denn da treten die meisten Abweichungen in der Messung auf, zum Beispiel beim GPS (Globale Positioning System) oder Altimeter (Barometer, das die Höhe mit Hilfe einer Höhenformel berechnet) - die beiden einfachsten und häufig genutzten Varianten. Kombiniert man beide vernünftig, so dass sie sich gegenseitig unterstützen, lässt sich ein genaueres Höhenmessgerät entwickeln.

So ist das Ziel der Arbeit ein Embedded-System zur Höhenmessung zu entwickeln, welches nahezu in jeder Situation, bei denen GPS und Altimeter ungenaue Resultate liefern würden, zuverlässige Messungen garantiert. Dabei wird nicht nur ein Hardware Konzept geschaffen, sondern auch eine dazu passende Software geschrieben. Ebenfalls Gegenstand der Arbeit ist das Testen unterschiedlicher Arten der Höhenmessung auf ihre Stärken und Schwächen. 

Um dieses Embedded-System bestehend aus einem GPS-Modul, einem Barometer und einem Beschleunigungssensor zu entwickeln, beruft sich der Autor auf ein Vorgehen, welches man unter den Namen “Sensor Fusion” kennt. Darunter versteht man das Kombinieren mehrerer Sensoren, mit dem Ziel bessere Resultate zu erhalten als mit einem einzelnen Sensor. Dieses Prinzip wird mit Hilfe zweier Methoden umgesetzt. Bei einer handelt es sich um das sogenannte Map-Matching (siehe Kapitel 3.1) und bei der anderen um den Kalman-Filter (siehe Kapitel 3.2). In Kapitel 4 werden die einzelnen Schritte im Rahmen einer Prozessdokumentation erläutert. 

%Noch Ändern
Zunächst werden einige theoretische Grundlagen zur Höhenmessung bekannt gegeben, die für das Verständnis der kommenden Probleme wichtig sind - ein solches Problem wäre z.B.: Warum gibt das GPS unpräzise Höhendaten in einem Parkhaus? Nebst dem GPS werden auch der Altimeter und der Beschleunigungssensor als weitere Varianten zur Höhenmessung erklärt. Danach werden die zwei Methoden - Map-Matching und Kalman-Filter - erläutert, sodass die benötigten Grundlagen für das nächste Kapitel gelegt sind. Dieses wäre die Prozessdokumentation des Produktes. Im Anschluss wird der Autor kurz auf die Resultate der Messungen eingehen und im Diskussionsteil interpretieren. Sprich, erklären warum die Resultate so aussehen und welche Bestandteile noch fehlen, damit es zu einem besseren Ergebnis kommen könnte.
